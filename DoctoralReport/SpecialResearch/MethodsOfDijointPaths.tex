\documentclass[specialreport]{subfiles}
\begin{document}
\section{素な経路選択問題の解決法}
前節で述べた素な経路選択の問題はメンガーの定理よりグラフの次数に等しい素な経路を構築する問題になる.
また, 同問題は最大流アルゴリズムを活用することで解くことが可能である. しかし, 最大流アルゴリズムは計算量が
$O(|V||E|)$となりグラフの頂点数と変数の積に等しいため実用的ではない. 
そして多くの場合グラフにおける素な経路選択問題はグラフの次数$n$に対する多項式時間のアルゴリズムで解決することを
目指している.

提案された相互結合網における素な経路選択を解決するにはグラフの特徴を性格に把握し活用する必要がある.
そして, 解決策の多くは以次に示すの手法を活用している.
\subsection{グラフの対称性}
提案されたグラフは全て対称グラフである. そのため任意の2頂点間での素な経

\subsection{グラフの再帰性}

\subsection{グラフ内部のリング構造}


\end{document}