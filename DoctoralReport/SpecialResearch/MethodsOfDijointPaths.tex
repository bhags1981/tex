\documentclass[specialreport]{subfiles}
\begin{document}
\section{素な経路選択問題の解決法}
前節で述べた素な経路選択の問題はメンガーの定理よりグラフの次数に等しい素な経路を構築する問題になる.
また, 同問題は最大流アルゴリズムを活用することで解くことが可能である. しかし, 最大流アルゴリズムは計算量が
$O(|V||E|)$となりグラフの頂点数と変数の積に等しいため実用的ではない. 
そして多くの場合グラフにおける素な経路選択問題はグラフの次数$n$に対する多項式時間のアルゴリズムで解決することを
目指している.

提案された相互結合網における素な経路選択を解決するにはグラフの特徴を性格に把握し活用する必要がある.
そして, 解決策の多くは以次に示すの手法を活用している.
\subsection{グラフの対称性}
提案されたグラフは全て対称グラフである. そのため任意の2頂点間での素な経

\subsection{グラフの再帰性}

\subsection{グラフ内部のリング構造}

\subsection{素な経路選択問題成果一覧}
\begin{table}[htb]
  \begin{center}
    \caption{素な経路選択問題成果一覧}
    \begin{tabular}{|c|c|c|c|} \hline
      グラフ&node-to-node&node-to-set&set-to-set\\ \hline 
      $Q_n$		&$\mathcal{O}(n^2) \cite{hq-n2n}$		&$\mathcal{O}(k^n)*1 \cite{hq-n2s}$		&$\mathcal{O}(n^2) 3$  \cite{hq-s2s}\\ \hline
      $S_n$		&$\mathcal{O}(n^2) \cite{star-n2s-first}$	&$\mathcal{O}(n^2) \cite{star-n2s} $		&$\mathcal{O}(n^2) \cite{star-n2s-first}$ \\ \hline
      $P_n$		&$\mathcal{O}(n^3) \cite{pan-n2n} $		&$\mathcal{O}(n^5) \cite{pan-n2s}$		&$\mathcal{O}(n^2)*2 \cite{pan-s2s}$\\ \hline
      $BP_n$	&$\mathcal{O}(n^3) \cite{bpn-n2n}$		&$\mathcal{O}(n^5) \cite{bpn-n2s}$		&$\mathcal{O}(n^4) \cite{bpn-s2s}$\\ \hline
      $R_n$		&??\cite{rot-n2n} 					&$\mathcal{O}(n^5) \cite{rot-n2s}$		&Open(2017.01.19) \\ \hline
      $BR_n$	&$\mathcal{O}(n^3)\cite{birot-n2n}$		&$\mathcal{O}(n^5) \cite{birot-n2s}$		&Open(2017.01.19)\\ \hline
      $T_n$		&$\mathcal{O}(n^7)\cite{tp-n2n}$		&$\mathcal{O}(n^6) \cite{tp-n2s2}$		&Open(2017.01.19)\\ \hline
      $S_n$		&$\mathcal{O}(n^7)\cite{ssr-n2n}$		&$\mathcal{O}(n^6) \cite{ssr-n2s}$		&Open(2017.01.19)\\ \hline
      $B_n$		&$\mathcal{O}(n^4) \cite{bs-n2n}$		&$\mathcal{O}(n^5) \cite{bs-n2s}$		&Open(2017.01.19)\\ \hline
    \end{tabular}
    \label{tab:intergrationratio}
  \end{center}
\end{table}


\begin{thebibliography}{9}
\bibitem{top500}Top 500, https://www.top500.org/lists/2016/11/, Jan. 2017.
\bibitem{kcomputer}Mitsuo Yokokawa, et al. "The K computer: Japanese next-generation supercomputer development project, " {\it Low Power Electronics and Design (ISLPED) 2011 International Symposium on. IEEE}, 2011.
\bibitem{oakforest}Oakforest-PACS スーパーコンピュータシステム, http://www.cc.u-tokyo.ac.jp/system/ofp/, Jan. 2017.
\bibitem{diskfailures}Disk failures in the real world: What does an MTTF of 1,000,000 hours mean to you?, https://www.usenix.org/legacy/events/fast07/tech/schroeder/schroeder\_html/index.html, Jan.2017.
\bibitem{hq-n2n} Y. Saad and M. H. Schultz, "Topological properties of hypercubes,"  {\it IEEE Transactions on Computers}, vol. 37, no. 7, pp. 867-872, July 1988. 
\bibitem{hq-n2s} Bossard, Antoine, and Keiichi Kaneko, “Time Optimal Node-to-Set Disjoint Paths Routing in Hypercubes,” {\it Journal of Information Science and Engineering}, Vol. 30, No. 4, pp. 1087-1093, July 2014.
\bibitem{hq-s2s} Gu, Qian Ping, Satoshi Okawa, and Shietung Peng. "Efficient Algorithms for Disjoint Paths in Hypercubes and Star Networks," {\it 数理解析研究所講究録} 871 (1994): 105-111. 
\bibitem{star-routing} Sanguthevar Rajasekaran and David S. L. Wei, "Selection, Routing, and Sorting on the Star Graph,	"  {\it Journal of Parallel and Distributed Computing}, {vol.42},{no.2},pp225-233, 1997.
\bibitem{star-optimal-prefix-computation}S.G. Akl and K. Qiu, "Data Communication and Computational Geometry on the Star and Pancake Interconnection Networks," {\it Parallel and Distributed Processing, 1991. Proceedings of the Third IEEE Symposium}, pp415-422,1991.
\bibitem{star-n2n} Qian-Ping Gu and Shietung Peng , "Node-to-node cluster fault tolerant routing in star graphs," {\it Information Processing Letters}, {vol.56},pp29-35, 1995.
\bibitem{star-n2s-first} M. Dietzfelbinger, S. Madhavapeddy, and I.H. Sudborough, "Three disjoint path paradigms in star networks," {\it Proc. IEEE SPDP}, pp.400-406, 1991.
\bibitem{star-n2s} Qian-Ping Gu and Shietung Peng , "Node-to-set disjoint paths problem in star graphs,"  {\it Information Processing Letters}, {vol.62},pp201-207, 1996.
\bibitem{pan-n2n} , Keiichi Kaneko, and Yasuto Suzuki, "An Algorithm for Node-Disjoint Paths in Pancake Graphs," {\it IEICE Trans. Information and Systems}, Vol. E86-D, No. 3, pp. 610-615, Mar. 2003
\bibitem{pan-n2s} Keiichi Kaneko, and Yasuto Suzuki, "Node-to-Set Disjoint Paths Problem in Pancake Graphs," {\it IEICE Trans. Information and Systems}, Vol. E86-D, No. 9, pp. 1628-1633, Sept. 2003.
\bibitem{pan-s2s} Peng, Shietung, and Keiichi Kaneko. "Set-to-set disjoint paths routing in pancake graphs," it{Proceedings of the IASTED International Conference on Parallel and Distributed Computing and Systems.} 2006.
\bibitem{bpn-n2n} Sawada, Naoki, Yasuto Suzuki, and Keiichi Kaneko. "Container problem in burnt pancake graphs." {\it International Symposium on Parallel and Distributed Processing and Applications Springer Berlin Heidelberg}, 2005.
\bibitem{bpn-n2s} Kaneko, Keiichi. "An algorithm for node-to-set disjoint paths problem in burnt pancake graphs." {\it IEICE TRANSACTIONS on Information and Systems} 86.12 (2003): 2588-2594.
\bibitem{bpn-s2s} Iwasawa, Nagateru, Antoine Bossard, and Keiichi Kaneko. "Set-to-Set Disjoint Path Routing Algorithm in Burnt Pancake Graphs." Proceedings of the ISCA 26th International Conference on Computers and their Applications. pp. 21-26. 2011.
\bibitem{rot-n2n} Suzuki, Yasuto, and Keiichi Kaneko: "An Algorithm for Node-Disjoint Paths in Rotator Graphs," {\it Proceedings of the Third International Conference on Parallel and Distributed Computing, Applications and Technologies}, pp. 151-158, Kanazawa, Japan, Sept. 4-6, 2002.
\bibitem{rot-n2s} Kaneko, Keiichi, and Yasuto Suzuki. "An algorithm for node-to-set disjoint paths problem in rotator graphs." {\it IEICE TRANSACTIONS on Information and Systems} 84.9 (2001): 1155-1163.
\bibitem{birot-n2n} Kaneko, Keiichi. "Internally-disjoint paths problem in bi-rotator graphs." {\it IEICE TRANSACTIONS on Information and Systems} 88.7 (2005): 1678-1684.
\bibitem{birot-n2s} Kaneko, Keiichi. "An algorithm for node-to-set disjoint paths problem in bi-rotator graphs." {\it IEICE transactions on information and systems} 89.2 (2006): 647-653.
\bibitem{tp-n2n}S. Latifi and P. K. Srimani, "Transposition networks as a class of fault-tolerant robust networks,'" {\it IEEE Transactions on Computers},{vol.45, no.2}, pp230-238, 1996.
\bibitem{tp-n2s}Y. Suzuki, K. Kaneko and M. Nakamori, "Node-disjoint paths in a transposition graph," {\it Proceedings of the International Conference on Parallel and Distributed, Processing Techniques and Applications},pp. 298-304,  2004. 
\bibitem{tp-n2s2}Satoshi Fujita, "Polynomial time algorithm for constructing vertex-disjoint paths in transposition graphs," {\it Networks}, Volume 56,  pp. 149-157, September 2010. 
\bibitem{ssr-n2n}Suzuki, Yasuto, Keiichi Kaneko, and Mario Nakamori. "Container problem in substring reversal graphs." {\it Parallel Architectures, Algorithms and Networks, 2004. Proceedings. 7th International Symposium on. IEEE}, 2004.
\bibitem{ssr-n2s}Jung, Sinyu, and Keiichi Kaneko. "Node-to-set disjoint paths in substring reversal graphs." {\it Computer Science and Software Engineering (JCSSE), 2011 Eighth International Joint Conference on. IEEE}, 2011.
\bibitem{bs-n2n} Suzuki, Yasuto, and Keiichi Kaneko. "The container problem in bubble-sort graphs." {\it IEICE transactions on information and systems} 91.4 (2008): 1003-1009.
\bibitem{bs-n2s} 鈴木康斗, and 金子敬一. "バブルソートグラフにおける素な経路選択アルゴリズム." {\it 電子情報通信学会論文誌} D 88.4 (2005): 751-756.
\bibitem{hs-s2s}Bossard, Antoine. "A Set-to-set Disjoint Paths Routing Algorithm in Hyper-star Graphs." {\it IJ Comput. Appl}. 21.1 (2014): 76-82.
\end{thebibliography}


\end{document}
